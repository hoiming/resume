\documentclass{resume}
\usepackage{zh_CN-Adobefonts_external} 
\usepackage{linespacing_fix}
\usepackage{cite}
\begin{document}
\pagenumbering{gobble}



%***"%"后面的所有内容是注释而非代码,不会输出到最后的PDF中
%***使用本模板,只需要参照输出的PDF,在本文档的相应位置做简单替换即可
%***修改之后,输出更新后的PDF,只需要点击Overleaf中的“Recompile”按钮即可
%**********************************姓名********************************************
\name{梁海明}
%**********************************联系信息****************************************
%第一个括号里写手机号,第二个写邮箱
\contactInfo{(+86) 15626001565}{haiming\_liang@hotmail.com}
%**********************************其他信息****************************************
%在大括号内填写其他信息,最多填写4个,但是如果选择不填信息,
%那么大括号必须空着不写,而不能删除大括号。
%\otherInfo后面的四个大括号里的所有信息都会在一行输出
%如果想要写两行,那就用两次这个指令(\otherInfo{}{}{}{})即可
\otherInfo{性别:男}{籍贯:广州}{}{}
%*********************************照片**********************************************
%照片需要放到images文件夹下,名字必须是you.jpg,如果不需要照片可以不添加此行命令
%0.15的意思是,照片的宽度是页面宽度的0.15倍,调整大小,避免遮挡文字

%**********************************正文**********************************************


%***大标题,下面有横线做分割
%***一般的标题有:教育背景,实习(项目)经历,工作经历,自我评价,求职意向,等等
\section{教育背景}


%***********一行子标题**************
%***第一个大括号里的内容向左对齐,第二个大括号里的内容向右对齐
%***\textbf{}括号里的字是粗体,\textit{}括号里的字是斜体
\datedsubsection{\textbf{华南师范大学},软件工程,\textit{硕士}}{2017.09 - 2019.07}


%***********列举*********************
%***可添加多个\item,得到多个列举项,类似的也可以用\textbf{}、\textit{}做强调



\datedsubsection{\textbf{华南农业大学},软件工程,\textit{本科}}{2008.09 - 2012.07}

\section{项目经验}
\datedsubsection{\textbf{基于Django和React的待办事项管理系统}}{2023.02至今}
\begin{itemize}[parsep=0.5ex]
  \item 后端使用Django 4和rest\_framework开发API,使用Model管理数据库。
  \item 前端使用React和react-bootstrap开发页面,实现待办事项的增删改查。
  \item 项目使用uwsgi和nginx,部署在阿里云服务器上,地址是:http://120.78.90.122:3000/
\end{itemize}
\datedsubsection{\textbf{基于Python的服务器依赖统一管理系统}}{2022.07至今}
\begin{itemize}[parsep=0.5ex]
  \item 基于Python2.7的BaseHTTPServer的依赖统一管理系统,使用glob、re和ConfigParser管理服务器依赖文件的版本。
  \item 使用urlparse分析URL中的查询参数,决定是否需要更新依赖版本。
  \item 使用了测试驱动开发,先编写了40个测试用例,对应10个API端点,先让测试运行失败,再开发代码,再运行测试用例,直至测试用例全部通过。
\end{itemize}
\datedsubsection{\textbf{基于Node.js的基础架构运维系统}}{2022.07至今}
\begin{itemize}[parsep=0.5ex]
  \item 基于Node.js和MongoDB的基础架构运维系统,使用Node.js和Feathersjs开发Restful API,提供前端使用。
  \item 开发基于Jenkins和Ansible的自动化部署Pipeline。
  \item 应用了Mocha作为单元测试框架,使用Git管理代码。
\end{itemize}

\section{职业经历}

\datedsubsection{\textbf{印孚瑟斯技术有限公司},Node.js开发工程师\hspace{0.7cm}Python开发工程师}{2022.07至今}
\begin{itemize}[parsep=0.5ex]
  \item 外包到汇丰软件(广东)有限公司,作为基础架构团队的一员,负责开发Node.js API操作MongoDB,创建Jenkins 流水线模板。
  \item 负责开发Shell脚本进行消息队列运行状态检查,维护Shell脚本。
  \item 开发Jenkins流水线和开发Ansible playbook对API服务进行自动化部署,避免了登录服务器手工更新服务。
\end{itemize}

\datedsubsection{\textbf{广发证券股份有限公司},Node.js开发工程师\hspace{0.7cm}Python测试开发工程师}{2021.08 - 2022.06}
\begin{itemize}[parsep=0.5ex]
  \item 使用Python开发对区块链存证平台的自动化测试,使用unittest编写测试用例,使用requests发送HTTP请求。
  \item 使用Java开发基于Springboot+Mybatis+Redis+MySQL的区块链存证管理平台的文件管理模块,开发了增加功能和查询功能。
  \item 开发Node.js API,提供给前端使用,实现了权限集中管理平台的后端服务,并负责部署测试环境,编写单元测试和集成测试,使用JIRA作为Bug管理平台,使用Fiddler抓包调试HTTP请求。参与项目并进行了多次项目迭代,成功实现了需求。
\end{itemize}

\datedsubsection{\textbf{中软国际},测试开发工程师}{2019.07 - 2021.07}
\begin{itemize}[parsep=0.5ex]
  \item 作为测试开发工程师,我被外派到汇丰软件(广东)有限公司,在敏捷团队中,我根据JSON Schema和RAML文件以及需求文档,使用Java语言开发并执行自动化测试用例。使用JIRA跟踪和管控项目质量。生成测试报告,提交上级评审,针对评审结果修改用例。API实现了如期交付,软件质量得到保证。
\end{itemize}

\datedsubsection{\textbf{文思海辉},VBA开发工程师}{2015.02 - 2016.09}
\begin{itemize}[parsep=0.5ex]
  \item 作为VBA开发工程师,我被外派到汇丰软件(广东)有限公司,在团队中,我根据需求文档,使用Excel VBA语言开发脚本,实现了金融交易的自动化和文档化。
\end{itemize}

\datedsubsection{\textbf{奔步电脑公司},C\#开发工程师}{2014.07 - 2015.02}
\begin{itemize}[parsep=0.5ex]
  \item 开发基于ASP.NET MVC的ERP系统。
  \item 维护基于Winform的珠宝销售管理系统。
\end{itemize}

\datedsubsection{\textbf{广州市诚毅软件开发有限公司},技术支持}{2012.06 - 2013.09}
\begin{itemize}[parsep=0.5ex]
  \item 对数字电视BOSS系统做技术支持,维护Oracle数据库中的业务数据。对业务员提供电话技术支持。协助项目组到客户现场进行系统上线。
\end{itemize}


\section{技能}

\begin{itemize}[parsep=0.5ex]
  \item Node.js, Javascript, Python, Django, rest\_framework
  \item React, Bootstrap
  \item Linux, Shell 脚本, Git/Github, Jenkins, Ansible
  \item Redis, MySQL
  \item AWS EC2, Lambda, DynamoDB
\end{itemize}


\section{语言}

\begin{itemize}[parsep=0.5ex]
  \item 普通话:流利
  \item 粤语:流利
  \item 英语:基本口语交流,读写顺畅。
\end{itemize}

\end{document}
